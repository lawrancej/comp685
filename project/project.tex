% !TEX TS-program = pdflatex
% !TEX encoding = UTF-8 Unicode

% This is a simple template for a LaTeX document using the "article" class.
% See "book", "report", "letter" for other types of document.

\documentclass[11pt]{article} % use larger type; default would be 10pt

\usepackage[utf8]{inputenc} % set input encoding (not needed with XeLaTeX)

%%% Examples of Article customizations
% These packages are optional, depending whether you want the features they provide.
% See the LaTeX Companion or other references for full information.

%%% PAGE DIMENSIONS
\usepackage{geometry} % to change the page dimensions
\geometry{a4paper} % or letterpaper (US) or a5paper or....
% \geometry{margin=2in} % for example, change the margins to 2 inches all round
% \geometry{landscape} % set up the page for landscape
%   read geometry.pdf for detailed page layout information

\usepackage{graphicx} % support the \includegraphics command and options

% \usepackage[parfill]{parskip} % Activate to begin paragraphs with an empty line rather than an indent

%%% PACKAGES
\usepackage{booktabs} % for much better looking tables
\usepackage{array} % for better arrays (eg matrices) in maths
\usepackage{paralist} % very flexible & customisable lists (eg. enumerate/itemize, etc.)
\usepackage{verbatim} % adds environment for commenting out blocks of text & for better verbatim
%\usepackage{subfig} % make it possible to include more than one captioned figure/table in a single float
% These packages are all incorporated in the memoir class to one degree or another...

%%% HEADERS & FOOTERS
%\usepackage{fancyhdr} % This should be set AFTER setting up the page geometry
%\pagestyle{fancy} % options: empty , plain , fancy
%\renewcommand{\headrulewidth}{0pt} % customise the layout...
%\lhead{}\chead{}\rhead{}
%\lfoot{}\cfoot{\thepage}\rfoot{}

%%% SECTION TITLE APPEARANCE
%\usepackage{sectsty}
%\allsectionsfont{\sffamily\mdseries\upshape} % (See the fntguide.pdf for font help)
% (This matches ConTeXt defaults)

%%% ToC (table of contents) APPEARANCE
%\usepackage[nottoc,notlof,notlot]{tocbibind} % Put the bibliography in the ToC
%\usepackage[titles,subfigure]{tocloft} % Alter the style of the Table of Contents
%\renewcommand{\cftsecfont}{\rmfamily\mdseries\upshape}
%\renewcommand{\cftsecpagefont}{\rmfamily\mdseries\upshape} % No bold!

%%% END Article customizations

%%% The "real" document content comes below...

\title{Project report}
\date{} % Activate to display a given date or no date (if empty),
         % otherwise the current date is printed 

\begin{document}
\maketitle

Structure the project report like this.

\section{Title (Name the project)}
Name the team members

\section{Abstract}
Overview what the team accomplished (learned) in one paragraph.

\begin{itemize}
\item What was the problem?
\item Why is the problem a problem?
\item What did the team do?
\item How did this address the problem?
\end{itemize}

\section{Introduction}
Describe the project, incorporating material from the project proposal.

\begin{itemize}
\item Describe the project.
\item Why this is a good/interesting project from the standpoint of usability?
\item Describe the usability problem you intend to solve.
\item Who are the target users?
\item Who used the prototypes?
\end{itemize}

\section{Related work}
Describe systems similar to your prototype.
\begin{itemize}
\item What is the existing system called?
\item Who uses it?
\item What are the problems with, or limitations of that system, if any?
\item Cite each system.
\end{itemize}

\section{Evaluation}
\subsection{Users}
Describe the people, objects, and environment in sufficient detail so that others can picture it.
\begin{itemize}
\item Who are the users?
\item What are the users doing? How?
\item What problems do the users encounter?
\item What objects did the users need or use?
\item What is the environment or space like?
\end{itemize}

\subsection{Method}
Describe how you planned to evaluate your prototypes.
\begin{itemize}
\item What questions did you ask? What tasks did the users perform?
\item How did you answer these questions? (Interview? Task analysis? Cognitive walkthrough?)
\item Describe the evaluation method. (E.g., task analysis, cognitive walkthrough, card sorting)
\item Why is this method appropriate to evaluate the prototype?
\end{itemize}

\section{Study}
Narrate the iterative progress of the prototype throughout the project.
Show each prototype (at least two different prototypes), the results of evaluting each prototype. Rinse and repeat.

\subsection{Prototype}
\begin{itemize}
\item Show the prototype. Scan in paper prototypes, include screenshots of high-fidelity prototypes.
\item Show a storyboard about how the prototype is used. 
\end{itemize}

\subsection{Results}
What did the team learn?
\begin{itemize}
\item Show the results (data). E.g., quote the user, show task analyses, click counts, etc.
\item Interpret the results. What did the evaluation reveal?
\item What insights did the team gain from observing users, analysing the task, etc.?
\item What problems did the user, analysis, or experimentation highlight in this prototype?
\item How did these findings influenced the next prototype. Explain.
\end{itemize}

\section{Discussion \& Conclusion}
The job of design is never complete. Where could the team take this?
\begin{itemize}
\item What questions remain unanswered?
\item What questions was the team unable to answer?
\item What could the team answer given sufficient resources?
\item What are some possible next steps?
\end{itemize}

\section{References}
Cite relevant books, chapters, papers, and web sites.
Include references here.

\end{document}
